%%%%%%%%%%%%%%%%%%%%%%%%%%%%%%%%%%%%%%%%%%%%%%%%%%%%%%%%%%%%%%%%%%%%
%%%%%%%%%%%%%%%%%%%%%%%%%%%%%%%%%%%%%%%%%%%%%%%%%%%%%%%%%%%%%%%%%%%%
%%                                                                %%
%% An example for writting your thesis using LaTeX                %%
%% Original version by Luis Costa,  changes by Perttu Puska       %%
%% Support for Swedish added 15092014                             %%
%%                                                                %%
%% This example consists of the files                             %%
%%         thesistemplate.tex (versio 2.01)                       %%
%%         opinnaytepohja.tex (versio 2.01) (for text in Finnish) %%
%%         aaltothesis.cls (versio 2.01)                          %%
%%         kuva1.eps                                              %%
%%         kuva2.eps                                              %%
%%         kuva1.pdf                                              %%
%%         kuva2.pdf                                              %%
%%                                                                %%
%%                                                                %%
%% Typeset either with                                            %%
%% latex:                                                         %%
%%             $ latex opinnaytepohja                             %%
%%             $ latex opinnaytepohja                             %%
%%                                                                %%
%%   Result is the file opinnayte.dvi, which                      %%
%%   is converted to ps format as follows:                        %%
%%                                                                %%
%%             $ dvips opinnaytepohja -o                          %%
%%                                                                %%
%%   and then to pdf as follows:                                  %%
%%                                                                %%
%%             $ ps2pdf opinnaytepohja.ps                         %%
%%                                                                %%
%% Or                                                             %%
%% pdflatex:                                                      %%
%%             $ pdflatex opinnaytepohja                          %%
%%             $ pdflatex opinnaytepohja                          %%
%%                                                                %%
%%   Result is the file opinnaytepohja.pdf                        %%
%%                                                                %%
%% Explanatory comments in this example begin with                %%
%% the characters %%, and changes that the user can make          %%
%% with the character %                                           %%
%%                                                                %%
%%%%%%%%%%%%%%%%%%%%%%%%%%%%%%%%%%%%%%%%%%%%%%%%%%%%%%%%%%%%%%%%%%%%
%%%%%%%%%%%%%%%%%%%%%%%%%%%%%%%%%%%%%%%%%%%%%%%%%%%%%%%%%%%%%%%%%%%%

%% Uncomment one of these:
%% the 1st when using pdflatex, which directly typesets your document in
%% pdf (use jpg or pdf figures), or
%% the 2nd when producing a ps file (use eps figures, don't use ps figures!).
\documentclass[english,12pt,a4paper,pdftex,sci,utf8]{aaltothesis}
%\documentclass[english,12pt,a4paper,dvips]{aaltothesis}

%% To the \documentclass above
%% specify your school: arts, biz, chem, elec, eng, sci
%% specify the character encoding scheme used by your editor: utf8, latin1

%% Use one of these if you write in Finnish (see the Finnish template):
%%
%\documentclass[finnish,12pt,a4paper,pdftex,elec,utf8]{aaltothesis}
%\documentclass[finnish,12pt,a4paper,dvips]{aaltothesis}

\usepackage{graphicx}

%% Use this if you write hard core mathematics, these are usually needed
\usepackage{amsfonts,amssymb,amsbsy}

%% Use the macros in this package to change how the hyperref package below 
%% typesets its hypertext -- hyperlink colour, font, etc. See the package
%% documentation. It also defines the \url macro, so use the package when 
%% not using the hyperref package.
%%
%\usepackage{url}

%% Use this if you want to get links and nice output. Works well with pdflatex.
\usepackage{hyperref}

\usepackage{subfigure}

\hypersetup{pdfpagemode=UseNone, pdfstartview=FitH,
  colorlinks=true,urlcolor=red,linkcolor=blue,citecolor=black,
  pdftitle={Default Title, Modify},pdfauthor={Yangjun Wang},
  pdfkeywords={Modify keywords}}


%% All that is printed on paper starts here
\begin{document}

%% Change the school field to specify your school if the automatically 
%% set name is wrong
% \university{aalto-yliopisto}
% \university{aalto University}
% \school{Sähkötekniikan korkeakoulu}
% \school{School of Science}

%% Only for B.Sc. thesis: Choose your degree programme. 
%\degreeprogram{Electronics and electrical engineering}
%%

%% ONLY FOR M.Sc. AND LICENTIATE THESIS: Specify your department,
%% professorship and professorship code. 
%%
\department{Department of Information and Computer Science}
\professorship{Data Communication Software}
%%

%% Valitse yksi näistä kolmesta
%%
%% Choose one of these:
%\univdegree{BSc}
\univdegree{MSc}
%\univdegree{Lic}

%% Your own name (should be self explanatory...)
\author{Yangjun Wang}

%% Your thesis title comes here and again before a possible abstract in
%% Finnish or Swedish . If the title is very long and latex does an
%% unsatisfactory job of breaking the lines, you will have to force a
%% linebreak with the \\ control character. 
%% Do not hyphenate titles.
%% 
\thesistitle{Benchmark Stream Processing Systems}

\place{Espoo}

%% For B.Sc. thesis use the date when you present your thesis. 
%% 
%% Kandidaatintyön päivämäärä on sen esityspäivämäärä! 
\date{28.12.2015}

%% B.Sc. or M.Sc. thesis supervisor 
%% Note the "\" after the comma. This forces the following space to be 
%% a normal interword space, not the space that starts a new sentence. 
%% This is done because the fullstop isn't the end of the sentence that
%% should be followed by a slightly longer space but is to be followed
%% by a regular space.
%%
\supervisor{Assoc. Prof.\ Aristides Gionis} %{Prof.\ Pirjo Professori}

%% B.Sc. or M.Sc. thesis advisors(s). You can give upto two advisors in
%% this template. Check with your supervisor how many official advisors
%% you can have.
%%
%\advisor{Prof.\ Pirjo Professori}
\advisor{D.Sc.\ Gianmarco De Francisci Morales}
%\advisor{M.Sc.\ Polli Pohjaaja}

%% Aalto logo: syntax:
%% \uselogo{aaltoRed|aaltoBlue|aaltoYellow|aaltoGray|aaltoGrayScale}{?|!|''}
%%
%% Logo language is set to be the same as the document language.
%% Logon kieli on sama kuin dokumentin kieli
%%
\uselogo{aaltoBlue}{''}

%% Create the coverpage
%%
\makecoverpage


%% Note that when writting your master's thesis in English, place
%% the English abstract first followed by the possible Finnish abstract

%% English abstract.
%% All the information required in the abstract (your name, thesis title, etc.)
%% is used as specified above.
%% Specify keywords
%%
%% Kaikki tiivistelmässä tarvittava tieto (nimesi, työnnimi, jne.) käytetään
%% niin kuin se on yllä määritelty.
%% Avainsanat
%%
\keywords{Big Data, Stream, Benchmark, Storm, Flink, Spark}
%% Abstract text
\begin{abstractpage}[english]
  Batch processing technologies(Such as MapReduce, Hive, Pig) have matured and been widely used in the industry. These systems solved the issue processing big volumes of data successfully. However, first big data need to be collected and stored in a database or file system. Then it takes time to finish batch processing analysis job before get any results. While there are many cases that need analysed results from streaming data immediately. The demand for processing real time stream data is increasing a lot these days. A big data architecture contains several parts. Often, masses of structured and semi-structured historical data are stored in Hadoop (Volume + Variety). On the other side, stream processing is used for fast data requirements (Velocity + Variety)\cite{GameChanger}. Several streaming processing systems are implemented and widely adopted, such as Apache Storm, JStorm, Apache Spark, IBM InfoSphere Streams and Apache Flink. They all support real-time stream processing, high scalability, and awesome monitoring. How to evaluate a real time stream processing system before choosing it to use in production development is a open question.  Before these real time stream processing systems are implemented, Michael demonstrated the 8 requirements\cite{8requirements} of real-time stream processing, which gives us a standard to evaluate whether a real time stream processing system satisfies these requirements.  A very common and traditional approach to verify whether the performance of a system meets the requirements is benchmarking. Published benchmarking results from industry standard benchmark systems could help users compare products and understand features of a system easily.  
\end{abstractpage}

%% Force a new page so that the possible English abstract starts on a new page
%%
%% Pakotetaan uusi sivu varmuuden vuoksi, jotta 
%% mahdollinen suomenkielinen ja englanninkielinen tiivistelmä
%% eivät tule vahingossakaan samalle sivulle
\newpage
%

%% Preface
%%
%% Esipuhe 
\mysection{Acknowledgements}
%\mysection{Esipuhe}
I want to thank ProfessorAristides Gionis
and my advisor Gianmarco De Francisci Morales for their 
good guidance.\\

\vspace{5cm}
Otaniemi, 16.1.2015

\vspace{5mm}
{\hfill Yangjun Wang \hspace{1cm}}

%% Force new page after preface
%%
%% Pakotetaan varmuuden vuoksi esipuheen jälkeinen osa
%% alkamaan uudelta sivulta
\newpage


%% Table of contents. 
\thesistableofcontents


%% Symbols and abbreviations
%%% Symbols and abbreviations
\addcontentsline{toc}{chapter}{Abbreviations and Acronyms}
\chapter*{Abbreviations and Acronyms}

\section*{Symbols}

\begin{tabular}{ll}
$\mathbf{B}$  & magnetic flux density  \\
$c$              & speed of light in vacuum $\approx 3\times10^8$ [m/s]\\
$\omega_{\mathrm{D}}$    & Debye frequency \\
$\omega_{\mathrm{latt}}$ & average phonon frequency of lattice \\
$\uparrow$       & electron spin direction up\\
$\downarrow$     & electron spin direction down
\end{tabular}

\section*{Operators}

\begin{tabular}{ll}
$\nabla \times \mathbf{A}$              & curl of vectorin $\mathbf{A}$\\
$\displaystyle\frac{\mbox{d}}{\mbox{d} t}$ & derivative with respect to 
variable $t$\\[3mm]
$\displaystyle\frac{\partial}{\partial t}$  & partial derivative with respect 
to variable $t$ \\[3mm]
$\sum_i $                       & sum over index $i$\\
$\mathbf{A} \cdot \mathbf{B}$    & dot product of vectors $\mathbf{A}$ and 
$\mathbf{B}$
\end{tabular}

\section*{Abbreviations}

\begin{tabular}{ll}
DFS	   	& Distribute File System \\
HDFS         & Hadoop Distribute File System \\
GFS		 & Google File System \\
YCSB	 & Yahoo Cloud Serving Benchmark
\end{tabular}
\clearpage


%% Tweaks the page numbering to meet the requirement of the thesis format:
%% Begin the pagenumbering in Arabian numerals (and leave the first page
%% of the text body empty, see \thispagestyle{empty} below).
%% Additionally, force the actual text to begin on a new page with the 
%% \clearpage command.
%% \clearpage is similar to \newpage, but it also flushes the floats (figures
%% and tables).
%% There is no need to change these
%%
\cleardoublepage
\storeinipagenumber
\pagenumbering{arabic}
\setcounter{page}{1}

\chapter{Introduction}
\label{chapter:intro}

Along with the rapid development of information technology, the speed of data generation increases dramatically. To process and analysis such large amount of data, the so-called Big Data,  cloud computing technologies get a quick development,  especially after these two papers related to MapReduce and BigTable published by Google  \cite{chang2006bigtable, dean2008mapreduce}.

\section{Stream Processing and Evaluation}
\label{section:big_data_analytics}
In theory, Big Data don't only mean "big" \textbf{v}olume. Besides volume, Big Data still have two other properties: \textbf{v}elocity and \textbf{v}ariety \cite{doug2001data}. Velocity means the amount of data is growing at high velocity. Variety refers to the various data formats.  They are called three \textbf{\textit{V}}s of Big Data.  When deal with Big Data, there are two types of processing model, batch processing and stream processing. A big data architecture contains several parts. For batch processing, masses of structured and semi-structured historical data are stored in HDFS (\textbf{V}olume + \textbf{V}ariety). On the other side, stream processing is used for fast data requirements (\textbf{V}elocity + \textbf{V}ariety)\cite{GameChanger}.

Batch processing is generally more concerned with throughput than latency of individual components of the computation. In batch processing, data is collected and stored in file system. When the size of data reaches a tradeoff, batch jobs could be configured to run without manual intervention, trained against entire dataset at scale in order to produce output in the form of computational analyses and data files. Because of time consume in data collection and processing stage, depending on the size of the data being processed and the computational power of the system, output can be delayed significantly. Generally, latency could be range from minutes to hours.

Streaming processing is required for many practical cases which need analysed results from streaming data in a very short latency. For example, a online shopping website would want give a customer accurate recommendations as soon as possible after he/she scan the website for a while.  Several streaming processing systems are implemented and widely adopted, such as Apache Storm, Apache Spark, IBM InfoSphere Streams and Apache Flink. They all support real-time stream processing, high scalability, and awesome monitoring. 

How to evaluate a real time stream processing system before choosing it to use in production development is a open question.  Before these real time stream processing systems are implemented, Michael demonstrated the 8 requirements\cite{8requirements} of real-time stream processing, which gives us a standard to evaluate whether a real time stream processing system satisfies these requirements.  A very common and traditional approach to verify whether the performance of a system meets the requirements is benchmarking. Published benchmarking results from industry standard benchmark systems could help users compare products and understand features of a system easily. In this thesis, we introduce a benchmark framework called StreamBench to facilitate performance comparisons of stream processing systems. 

\section{Structure of the Thesis}
\label{section:structure} 
The main topic of this thesis is stream processing systems benchmark. First, big data and cloud computing technology background is introduced in Chapter 2. Chapter 3 presents architecture and main features of three widely used stream processing systems. In Chapter 4, we demonstrate the design of our benchmark framework -- StreamBench, including the whole architecture, test data generator and extensibility of StreamBench. Experiments and results are discussed in Chapter 5. At last, conclusions are given in Chapter 6.









\clearpage

\chapter{Background}
Background knowledge of StreamBench which includes Big Data, Cloud Computing and widely accepted benchmark systems. 

\section{Cloud Computing} 
Concept of Cloud Computing and how Cloud Computing solves Big Data issues

\subsection{Parallel Computing}
\subsection{Computing Cluster}
\subsection{Batch Processing and Stream Processing}

\section{Apache Hadoop}
Introduce Apache Hadoop and several important modules

\subsection{MapReduce}
\subsection{Hadoop Distribution File Systems}
\subsection{YARN}
\subsection{Zookeeper}


\section{Benchmark}
Describe benchmark systems of traditional database and cloud service systems. 
Demonstrate design and components of benchmark system
Traditional database management systems were evaluated with industry standard benchmarks like TPC-C\cite{TPC-C}, TPC-H\cite{TPC-H}. These have focused on simulating complete business computing environment where plenty of users execute business oriented ad-hoc queries that involve transactions, big table scan, join and aggregation. The queries and the data populating the database have been chosen to have broad industry-wide relevance. This benchmark illustrates decision support systems that examine large volumes of data, execute queries with a high degree of complexity, and give answers to critical business questions\cite{TPC-H}. The integrity of the data is verified during the process of the execution of the benchmark to check whether the DBMS corrupt the data. If the data is corrupted, the benchmark measurement is rejected entirely\cite{dey2014ycsb+t}. Benchmark systems for DBMS mature, with data and workloads simulating real common business use cases, they could evaluate performance of DBMS very well. Some other works were done related to specific business model. Linkbench\cite{LinkBench} benchmarks database systems which store "social network" data specifically. The workload of database operations are based on Facebook's production workload and the data is also generated in such a way that key properties of the data match the production social graph data in Facebook.  

\subsection{Traditional Database Benchmark}
\subsection{Cloud Service Benchmark}


\clearpage


\section{Stream Processing Platforms}
Introduce three widely used stream processing platforms, point out core concepts and key features 

\subsection{Apache Storm}
\subsubsection{Storm Architecture}
\subsubsection{Computing Model}

\subsection{Apache Flink}
\subsubsection{Flink Architecture}
\subsubsection{Memory Management}
\subsubsection{Flink Streaming}


\subsection{Apache Spark}
\subsubsection{ Resilient Distributed Datasets(RDDs)}
\subsubsection{Spark Streaming}


\clearpage

\chapter{Benchmark Design}
We developed a tool, called StreamBench, to execute benchmark workloads on stream processing systems. A key feature of StreamBench is extensibility, so that it could be extended not only to run new workloads but also to benchmark new stream processing systems. We have used StreamBench to measure the performance of several stream processing systems, as we report in the next chapter.  StreamBench is also available under an open source license, so that others may use and extend it, and contribute new workloads and stream processing system interfaces.

In this chapter, we describe the architecture of StreamBench and introduce more detail of main components of StreamBench. 

\section{Architecture}

\begin{figure}
  \begin{center}
  \subfigure{\includegraphics[scale=0.6]{images/benchmark_architecture}}
   \caption{StreamBench architecture}
   \label{fig:streambench_architecture}
  \end{center}
\end{figure}

The main component of StreamBench is a Java program for consuming data from partitioned kafka topic and runs workloads on stream processing cluster. In the program, there is a set of common APIs for stream processing which could be engined by stream processing systems. Currently we support these APIs on three stream processing systems: Storm, Flink and Spark Streaming. Several workloads are implemented by these common APIs. In StreamBench we implemented three workloads to benchmark performance of stream processing systems in different aspects. The architecture of StreamBench is shown in Figure~\ref{fig:streambench_architecture}. 

Except the core Java program, the architecture also includes three more components: Cluster Deploy, Data Generator and Statistic. Section ~\ref{chapter:environment_setup} describes how to use cluster deploy scripts to setup experiment environment. Data generators generate test data for workloads and send it to kafka cluster that is demonstrated detailedly in Section~\ref{section:data_generator}. The Statistic component discussed in Section~\ref{section:log_statistic} includes experiment logging and performance statistic. 

\section{Experiment Environment Setup}
\label{chapter:environment_setup}

The operator system running on experiment nodes is Ubuntu 14.04 LTS. Benchmarked stream processing systems are Spark-1.5.1, Storm-0.10.0 and Flink-0.10.1. To enable checkpoint feature of Spark, Hadoop2.6(HDFS) is installed in compute cluster. Kafka 0.8.2.1 is running as distribute message system here. 

To deploy these software in compute cluster and kafka cluster automatically, we developed a set of python script. The prerequisites of using these scripts include internet access, ssh passwordless login between nodes in cluster and cluster configuration that describes which nodes are compute node or kafka node and where is the master node. The basic logic of deploy scripts is to download softwares online and install them, then replace configure files which are contained in a Github repository. For detail information of how to use cluster deploy scripts and configure of Storm, Flink, Spark and Kafka, please check this Github repository~\footnote{\url{https://github.com/wangyangjun/StreamBench}}.

\section{Workloads}
\label{section:workloads}

In StreamBench, a workload consists of a stream processing application and one or more kafka topics. The application consumes messages from kafka cluster and executes operations or transformations on the messages. We have developed 3 workloads to evaluate different aspects of a system's performance. Each workload contains a representative operation or feature of stream processing system that can be used to evaluate systems at one particular point in the performance space. We have not attempted to exhaustively examine the entire performance space. As StreamBench is open sourced, users could also defined their own workloads either by defining a new set of workload parameters, or if necessary by implement a new workload which is discussed detailedly in section~\ref{section:extensibility}.


\subsection{Basic Operators}

With the widespread use of computer technologies, there is increasing demand of processing unbounded, continuous input streams. In most cases, only basic operations need to be performed on the data streams such as \texttt{map}, \texttt{reduce}. One good sample is stream WordCount. WordCount is a very common sample application of Hadoop MapReduce that counts the number of occurrences of each word in a given input set.\cite{MapReduce} Similarly, many stream processing systems also support it as an sample application to count words in a  given input stream. Stream WordCount is implemented with basic operations which are supported by almost all stream processing systems. It means either the system has such operations by default or the operations could be implemented with provided built-in APIs. Other basic operations include \texttt{flatMap}, \texttt{mapToPair} and \texttt{filter} which are similar to \texttt{map} and could be implemented by specializing \texttt{map} if not supported by default. In StreamBench, there are a set of corresponding basic APIs defined. The pseudocode of WordCount implemented with these basic APIs could be abstracted as Algorithm~\ref{alg:word_count}.

\begin{algorithm}
\caption{WordCount}\label{euclid}
\label{alg:word_count}
\begin{algorithmic}[1]
\State $\textit{sentenceStream.flatMap(...)}$
\State \hspace{2.6cm} $\textit{.mapToPair(...)}$
\State \hspace{2.6cm} $\textit{.reduceByKey(...)}$
\State \hspace{2.6cm} $\textit{.updateStateByKey(...)}$

\end{algorithmic}
\end{algorithm}

One special case of the basic APIs is \texttt{updateStateByKey}. Only in Spark Streaming there is a corresponding built-in operation. As discussed in Section~\ref{section:spark}, the computing model of Spark Streaming is micro-batch which is different with that of other stream processing systems. The results of operation \texttt{reduceByKey} of WordCount running in Spark Streaming is word counts of one single micro batch data set. Operation \texttt{updateStateByKey} is used to accumulate word counts in Spark Streaming. Because the model of Flink and Storm is stream processing and accumulated word counts are returned from \texttt{reduceByKey} directly. Therefore, when implementing the API \texttt{updateStateByKey} with Flink and Storm engine, nothing need to do. The goal of this workload is to evaluate the performance of stream processing systems executing basic operations.  


\subsection{Join Operator}
 \label{sub:join_operator}

Besides the cases in which only basic operations are performed, another typical type of stream use case is processing joins over unbounded streams. For example, in a surveillance application, we may want to correlate cell phone traffic with email traffic. Theoretically unbounded memory is required to processing join over unbounded input streams, since every record in one infinite stream must be compared with every record in the other. Obviously, this is not practical.\cite{window-join} Since the memory of a machine is limited, we need restrict the number of records stored for each stream with a time window. 

A window join takes two key-value pair streams of tuple, say stream \textit{S1:} \texttt{(k, v1)} and stream \textit{S2:} \texttt{(k, v2)}, along with window sizes for both \textit{S1} and \textit{S2} as input. The output is a stream of tuple \texttt{(k, v1, v2)}. Assuming a sliding window join between stream \textit{S1} and stream \textit{S2}, a new tuple arrival from stream \textit{S1}, then a summary of steps to preform join is the following:

\begin{enumerate}
\item Scan window of stream \textit{S2} to find any tuple which has the same key with this new tuple and propagate the result;
\item 
\begin{enumerate}
\item Insert the new tuple into stream \textit{S1}'s window or
\item  invalidate target tuple in stream \textit{S2}'s window if found;
\end{enumerate}
\item Invalidate all expired tuples in stream \textit{S1}'s window.
\end{enumerate}

\begin{figure}
  \begin{center}
  \subfigure{\includegraphics[scale=0.6]{images/join}}
   \caption{Window join scenario}
   \label{fig:window_join}
  \end{center}
\end{figure}

In step 2, there are two options. In the case of at most one tuple which has the same key with the new tuple could be found in stream \textit{S2}, option (b) is executed. Otherwise, option (a) is performed. Every time new tuple arrives stream \textit{S1}, window of stream \textit{S2} need be scanned. That reduces the performance of join operator, especially when the window is big.

In the first case mentioned above, with a data structure named \texttt{cachedHashTable} there is another way to implement stream join. The tuples in the window of a stream are stored in a cached hash table. Each tuple is cached for window time and expired tuples are invalidated automatically. One of such a \texttt{cachedHashTable} could be found in Guava.\footnote{\url{http://docs.guava-libraries.googlecode.com/git/javadoc/com/google/common/cache/CacheBuilder.html}} Instead of scanning window of stream \textit{S2}, we could find tuple with the same key in \textit{S2} directly by calling \texttt{cachedHashTable.get(k)}. In theory, this implementation achieves better performance. 

Since Spark Streaming doesn't process tuples in a stream one by one, the join operator in Spark Streaming has different behaviours. In each batch interval, the RDD generated by stream1 will be joined with the RDD generated by stream2. For windowed streams, as long as slide durations of two windowed streams are the same, in each slide duration, the RDDs generated by two windowed streams will be joined. Because of this, window join in Spark Streaming could only make sure that a tuple in one stream will always be joined with corresponding tuple in the other stream that arrived earlier up to a configureable window time. Otherwise, repeat joined tuples would exist in generated RDDs of joined stream. As Figure~\ref{fig:spark_join_norepeat} shown, a tuple in Stream2 could be always joined with a corresponding tuple in Stream1 that arrived up to 2 seconds earlier. Since the slide duration of Stream2 is equal to its window size, no repeat joined tuple exists. On the other hand, it is possible that a tuple arrives earlier from Stream2 than the corresponding tuple in Stream1 couldn't be joined. Figure~\ref{fig:spark_join_repeat} exemplifies that there are tuples joined repeatedly  when slide duration of Stream2 is not equal to its window size.

\begin{figure}
  \begin{center}
  \subfigure{\includegraphics[scale=0.6]{images/spark_join_norepeat}}
   \caption{Spark Stream join without repeated tuple}
   \label{fig:spark_join_norepeat}
  \end{center}
\end{figure}


To evaluate performance of join operator in stream processing systems, we designed a workload called AdvClick which joins two streams in a online advertisement system. Every second there are a huge number of web pages opened which contain advertisement slots. A corresponding stream of shown advertisements is generated in the system and a record in the stream could be simply described as a tuple of \texttt{(id, shown time)}. Some of advertisements would be clicked by users and clicked advertisements is a stream which could be abstracted as a unbound tuples of \texttt{(id, clicked time)}. We assume that a record of advertisement shown always arrivers earlier than corresponding click record. Normally, if an advertisement is attractive to a user, it will be clicked in seconds or a few minutes after shown. We call such a click of an attractive advertisement valid click. To bill a customer, we need count all valid clicks regularly for advertisements of this customer. Which could be counted after joining stream \texttt{advertisement clicks} and stream \texttt{shown advertisements} with a configureable window time. 

\begin{figure}
  \begin{center}
  \subfigure{\includegraphics[scale=0.6]{images/spark_join_repeat}}
   \caption{Spark Stream join with repeated tuples}
   \label{fig:spark_join_repeat}
  \end{center}
\end{figure}

\subsection{Iterate Operator}

Iterative algorithms occur in many domains of data analysis, such as machine learning or graph analysis. Many stream data processing tasks require iterative sub-computations as well. To achieve these requirements, a data processing system should have the capacity to perform iterative processing on a real-time data stream. To achieve iterative sub-computations,  low-latency interactive access to results and consistent intermediate outputs, \citeauthor{murray2013naiad} introduced a computational model named timely dataflow that is based on a directed graph in which stateful vertices send and receive logically timestamped messages along directed edges. \cite{murray2013naiad} The dataflow graph may contain nested cycles and the timestamps reflect this structure in order to distinguish data that arise in different input epochs and loop iterations. With iterate operator, many stream processing systems already support such nested cycles in processing data flow graph. We designed a workload named StreamKMeans to evaluate iterate operator in stream processing systems.

KMeans is a clustering algorithm which aims to partition n points into k clusters in which each point belongs to the cluster with the nearest mean, serving as a prototype of the cluster.\cite{kmeans_wiki} Given an initial set of k means, the algorithm proceeds by alternating between two steps: \cite{mackay2003information}
\begin{description}
\item\textbf{Assignment step:} assign each point to the cluster whose mean yields the least within-cluster sum of squares.
\item \textbf{Update step:} Calculate the new means to be the centroids of the points in the new clusters.
\end{description}

The algorithm has converged when the assignments no longer change. We apply k-means algorithm on a stream of points with an iterate operator to update centroids.

Compared to clustering for data set, the clustering problem for the data stream domain is difficult because of two issues that are hard to address: (1) The quality of the clusters is poor when the data evolves considerably over time. (2) A data stream clustering algorithm requires much greater functionality in discovering and exploring clusters over different portions of the stream.\cite{aggarwal2003framework} Considering the main purpose of this workload is to evaluate iterative loop in stream data processing, we don't try to solve these issues here. Similarly, stream k-means also has two steps: assignment and update. The difference is each point in the stream only passes the application once and the application doesn't try to buffer points. As shown in Figure~\ref{fig:iterator_operator}, once a new centroid calculated, it will be broadcasted to assignment executors. 

 \begin{figure}
  \begin{center}
  \subfigure{\includegraphics[scale=0.6]{images/iterator_operator}}
   \caption{Stream k-means scenario}
   \label{fig:iterator_operator}
  \end{center}
\end{figure}

Spark executes data analysis pipeline using directed acyclic graph scheduler.  Nested cycle doesn't exist in the data pipeline graph. Therefore, this workload will not be used to benchmark Spark Streaming. Instead, a standalone version of k-means application is used to evaluate the performance of Spark Streaming.

\section{Data Generators}
\label{section:data_generator}
 A data generator is a program that produces and sends unbound records continuously to kafka cluster which are consumed by corresponding workload. For each workload, we designed one or several data generators with some parameters configureable which define the skew in record popularity, the size of records etc. These parameters could be changed to evaluate the performance of a system executing one workload on similar data streams with different properties. 

\subsection{WordCount}
\label{subsection:wordcount_generator}

Generators of workload WordCount produce unbound lists of sentences, each sentence consists of several words. The number of words in each sentence satisfies normal distribution with mean and variance configureable. Each word is a 5-digit zero-padded string of a binary integer, such as ``00001". There are two generators implemented with these integers satisfy two different distribution: uniform distribution and normal distribution.

\subsection{AdvClick}

As discussed in Section~\ref{sub:join_operator}, workload AdvClick joins stream \texttt{shown advertisements} and stream \texttt{advertisement clicks}. Each shown advertisement is a tuple consist of a universally unique identifier(\textbf{UUID}) and a timestamp. Each advertisement has a probability to be clicked. Then the data generator could be a multi-threads application with main thread producing advertisements and sub-threads producing clicks. The pseudocode of the main thread is shown as Algorithm \ref{alg:advclick_generator}. After a sub-thread starts, it sleeps for dalta time and then sends click record to corresponding kafka topic. 

\begin{algorithm}
\caption{AdvClick data generator}\label{euclid}
\label{alg:advclick_generator}
\begin{algorithmic}[1]
\State $\text{load } \textit{clickProbability} \text{ from configure file}$

\State $\textit{cachedThreadPool} \gets \text{new CachedThreadPool}$
\State $\textit{dataGenerator} \gets \text{new RandomDataGenerator}$ 
\State $\textit{producer} \gets \text{new KafkaProducer}$ 

\While{not interrupted}
\State $\textit{advId} \gets \text{new UUID}$ 
\State $\textit{timestamp} \gets \text{current timestamp}$ 
\State $\textit{producer.send(...)}$ 

\If {$\textit{generator.nextUniform(0,1)} < clickProbability$} 
\State $\textit{deltaTime} \gets \textit{generator.nextGaussian(...)}$ 
\State $\textit{cachedPool.submit(new ClickThread(advId, daltaTime))} $ 
\EndIf
\EndWhile
\end{algorithmic}
\end{algorithm}

\subsection{KMeans}

Stream k-means is a one-pass clustering algorithm for stream data. In this workload, it is used to cluster a unbound stream of points. First, a set of centroids are generated and wrote to a external file. Then the generator produces points according these centroids as Algorithm~\ref{alg:kmeans_generator}.

\begin{algorithm}
\caption{KMeans data generator}\label{euclid}
\label{alg:kmeans_generator}
\begin{algorithmic}[1]
\State $\text{load } \textit{covariances} \text{ from configure file}$
\State $\textit{means} \gets \text{original point}$
\State $\text{load } \textit{centroids} \text{ from external file}$

\State $\textit{producer} \gets \text{new KafkaProducer}$ 
\State $\textit{normalDistributon } \gets \text{ new NormalDistribution(means, converiances)}$

\While{not interrupted}
\State $\textit{centroid} \gets \text{pick a centroid from \textit{centroids} randomly}$ 
\State $\textit{point} \gets \textit{centroid+normalDistributon.sample()}$ 
\State $\textit{producer.send(point)}$ 

\EndWhile
\end{algorithmic}
\end{algorithm}


\section{Experiment Logging and Statistic}
\label{section:log_statistic}

For evaluating the performance, there are two performance measurement terms used in StreamBench that are latency and throughput. Latency is the required time from a record entering the system to some results produced after some actions performed on the record. In StreamBench, messaging system and stream processing system are combined together and treated as one single system. The latency is computed start from when a record is generated. As discussed in Section \ref{section:data_generator}, data is sent to kafka cluster immediately after generation. Figure~\ref{fig:latency} shows how latency computed in StreamBench. 

Throughput is the number of actions executed or results produced per unit of time. In the WordCount workload, throughput is computed as the number of words counted per seconds in the whole compute cluster. Joined clicked stream and the number of points processed per second are the throughput of workloads AdvClick and Stream KMeans respectively.

There is an inherent tradeoff between latency and throughput: on a given hardware setup, as the amount of load increases by increase the speed of data generation, the latency of individual records increases as well since there is more contention for disk, CPU, network, and so on. Computing latency start from records generated makes it easy to measure the highest throughput, since records couldn't produced in time will stay in kafka topics that increase latency dramatically. A stream processing system with better performance will achieve low latency and high throughput with fewer servers.

\begin{figure}
  \begin{center}
  \subfigure{\includegraphics[scale=0.5]{images/latency}}
   \caption{Latency}
   \label{fig:latency}
  \end{center}
\end{figure}

\section{Extensibility}
\label{section:extensibility}

One significant feature of StreamBench is extensibility. The component "Workloads" in Figure~\ref{fig:streambench_architecture} contains three predefined workloads discussed in Section~\ref{section:workloads} that are implemented with common stream processing APIs. First, with some configuration modification of a data generator, which allows user to vary the skew in record popularity, and the size and number of records. The performances of a workload processing data streams with different properties could be different a lot. Moreover, it is easy for developers to design and implement a new workload to benchmark some specific features of stream processing systems. This approach allows for introducing more complex stream processing logic, and exploring tradeoffs of new stream processing features; but involves greater effort compared to the former approach.

Besides implementing new workloads, StreamBench also could be extended to benchmark new stream processing systems by implement a set of common stream processing APIs. A few samples of APIs could be shown as following:
\begin{itemize}
\item \textbf{map}(MapFunction\textless \textbf{T}, \textbf{R}\textgreater fun, String componentId): map each record in a stream from type \textbf{T} to type \textbf{R}
\item \textbf{mapToPair}(MapPairFunction\textless \textbf{T, K, V}\textgreater fun, String componentId): map a  item stream\textbf{\textless T\textgreater } to a pair stream\textbf{\textless K, V\textgreater}
\item \textbf{reduceByKey}(ReduceFunction\textless \textbf{V}\textgreater fun, String componentId): called on a pair stream of (\textbf{K, V}) pairs, return a new pair stream of (\textbf{K, V}) pairs where the values for each key are aggregated using the given reduce function
\end{itemize}

These methods are quite simple, representing common data transformations. There are some other APIs like \texttt{filter()}, \texttt{flatMap()} and \texttt{join} which are also easily to implement and supported well by most stream processing systems. Despite its simplicity, this API maps well to the native APIs of many of the stream processing systems we examined.







\chapter{Experiment}
\label{chapter:experiment}

After StreamBench architecture and design of workloads are demonstrated, this chapter will draw our attention to experiments. Each experiment case is executed several times to get a stable result. In this chapter, we present experiment results of three selected stream processing systems running the workloads that are discussed in~\cref{section:workloads}. As illustrated in \cref{section:log_statistic}, two performance metrics that we are concerned with latency and throughput. For each workload, we compare the experiment results of different stream processing systems with visualization of these two metrics. 

\section{WordCount}
\label{section:wordcount_experiment}

First, we examine workload WordCount which aims to evaluate performance of stream processing systems performing basic operators. In order to check different performance metrics of stream processing systems, we performed the WordCount experiments in two different models: Offline model and Online model. Offline WordCount focuses on throughput and aims to find maximum throughput of this workload performed on each system. Offline means that the workload application consumes data that already exists in Kafka cluster. On the contrary, experiments of consuming continuous coming data is called Online model, which measures latency of stream processing with different throughputs. Moreover, we also made some modification to the original workload to evaluate pre-aggregation property of Storm.  As mentioned in~\cref{subsection:wordcount_generator}, for this workload, we designed two data generators to produce words that satisfy uniform and zipfian distributions. Comparison between experiment results of processing these two different data streams is also presented. 


\subsection{Offline WordCount}
\begin{figure}[t!]
  \begin{center}
  \subfigure[Skewed WordCount ]{\includegraphics[scale=0.27]{images/throughput_skewed}}
  ~
  \subfigure[Uniform WordCount ]{\includegraphics[scale=0.27]{images/throughput_uniform}}
   \caption{Throughput of Offline WordCount (words/second)}
   \label{fig:offline_throughput}
  \end{center}
\end{figure}

Since the computing model of Spark Streaming is micro-batch processing, existing data in Kafka cluster would be collected and processed as one single batch. The performance of processing one large batch with Spark Streaming is similar to a Spark batch job. There are already many works evaluating performance of Spark batch processing. Therefore, we skip experiments of Spark Streaming here. Figure~\ref{fig:offline_throughput} presents throughputs of Offline WordCount processing both skewed and uniform data on Storm and Flink clusters. It is obvious that the throughput of Flink is incomparably larger than Storm, tens of times higher. The throughput of Flink cluster dealing with uniform data stream is very high and reaches 2.83 million words per second, which is more than two times as large as throughput of performing skewed data. The corresponding ratio of Storm is around 1.25. The skewness of experiment data has greater influence of performance on Flink than Storm.

\begin{figure}[t!]
  \begin{center}
  \subfigure[Storm (ack enabled)]{\includegraphics[scale=0.25]{images/storm_throughput_scale}}
  ~
  \subfigure[Flink]{\includegraphics[scale=0.25]{images/flink_throughput_scale}}
   \caption{Throughput Scale Comparison of Offline WordCount}
   \label{fig:offline_throughput_scale}
  \end{center}
\end{figure}

The difference of performance between skewed data and uniform data indicates that the bottleneck of a cluster processing skewed data would be the node dealing with the data with highest frequency. To verify this assumption, we reduce the number of computing nodes from 8 to 4, and run these experiments. The experiment results are presented as Figure~\ref{fig:offline_throughput_scale}. The throughput of 8-nodes cluster of both systems dealing with uniform data is nearly two times as large as that of 4-nodes cluster. It means that the scalability of both systems is good. While processing skewed data, increasing the number of work nodes in a Flink cluster doesn't bring significant performance increase. Storm cluster gets about 58\% throughput improvement when increasing cluster from 4 nodes to 8 nodes. The result indicates that the assumption is correct in Flink, and the bottleneck of a storm cluster might be other factors.

The throughput of each work node in computing cluster is displayed in Figure~\ref{fig:worknodes_throughput}. Obviously, for both Storm and Flink, workloads processing uniform data achieve better balance than corresponding workloads dealing with skewed data. Either dealing with uniform data or skewed data, Storm achieves better workload balance than Flink. The experiment results also shows that Flink cluster with 4 compute nodes has better workload balance than clusters with 8 nodes.


\begin{figure}[t!]
  \begin{center}
   \subfigure{\includegraphics[scale=0.35]{images/storm4nodes_skewed_throughput}}
  ~
  \subfigure{\includegraphics[scale=0.35]{images/storm4nodes_uniform_throughput}}
  ~
  \subfigure{\includegraphics[scale=0.34]{images/storm_skewed_throughput}}
  ~
  \subfigure{\includegraphics[scale=0.34]{images/storm_uniform_throughput}}
  ~
    \subfigure{\includegraphics[scale=0.33]{images/flink4nodes_skewed_throughput}}
  ~
  \subfigure{\includegraphics[scale=0.33]{images/flink4nodes_uniform_throughput}}
  ~
  \subfigure{\includegraphics[scale=0.33]{images/flink_skewed_throughput}}
  ~
  \subfigure{\includegraphics[scale=0.33]{images/flink_uniform_throughput}}

   \caption{Throughput of work nodes (words/s)}
   \label{fig:worknodes_throughput}
  \end{center}
\end{figure}

\subsection{Online WordCount}
\label{subsec:online_wordcount}

Base on the experiment results of Offline WordCount, we perform experiments of Online WordCount on Storm and Flink at around half of the maximum achieved throughput of Offline WordCount respectively. In Online scenario, the stream processing application starts earlier than data generation. Which means data is processed as soon as possible after it is generated. As mentioned in \cref{section:log_statistic}, the latency is computed as spending time from a record generated to corresponding result computed.

% 99-th 521
\begin{figure}[t!]
  \begin{center}
  \subfigure[Records latency]{\label{fig:records}\includegraphics[scale=0.25]{images/latency_skewedwordcount}}
  \subfigure[Micro-batches latency]{\label{fig:batches}\includegraphics[scale=0.32]{images/spark_wordcount_latency}}
   \caption{Latency of Online WordCount}
   \label{fig:online_wordcount_latency}
  \end{center}
\end{figure}

In Spark Streaming, depending on the nature of the streaming computation, the batch interval used may have significant impact on the data rates that can be sustained by the application on a fixed set of cluster resources\footnote{\url{http://spark.apache.org/docs/1.5.1/streaming-programming-guide.html\#setting-the-right-batch-interval}}. Here, we perform the experiments with one second micro-batch interval and 10 seconds checkpoint interval which are the default configurations. Checkpointing is enabled because of a stateful transformation, \texttt{updateStateByKey} is used here to accumulate word counts.  Checkpointing is very time consuming due to writing information to a fault- tolerant storage system. Figure~\ref{fig:online_wordcount_latency}\subref{fig:batches} shows that the latency of micro-batches increasing and decreasing periodically because of checkpointing. A micro-batch is collected during one micro-batch interval, early records are buffered before the last record in the micro-batch arrives. In figure~\ref{fig:online_wordcount_latency}\subref{fig:batches}, buffer time of records in a micro-batch is not took in consideration. Before the computation of a micro-batch is finished, computation job of following micro-batches will not start. Therefore, the start time of computation job of a micro-batch would be delayed, this is indicated by ``Delay" in the figure. The throughput of experiment corresponding to Figure~\ref{fig:online_wordcount_latency}\subref{fig:batches} is 1.4M/s (million words per second) of skewed data. When the speed of data generation reaches 1.8M/s, the delay and latency increase infinitely with periodic decreasing.

Figure~\ref{fig:online_wordcount_latency}\subref{fig:records} shows the latency of Online WordCount performing skewed data. Storm with ack enabled achieves a median latency of 10 milliseconds, and a 95-th percentile latency of 201 milliseconds, meaning that 95\% of all latencies were below 201 milliseconds. Flink has a higher median latency (39 milliseconds), and a similar 95-th percentile latency of 217 milliseconds. Since the records  in a micro-batch are buffered up to batch interval time, the buffer time are also counted into the latency according to our latency computational method present in Figure~\ref{fig:latency}. For example, median latency of Spark Streaming is equal to the sum of median latency of micro-batches and half of micro-batch interval. Obviously, the latency of Spark Streaming is much higher than that of others.


As mentioned in \cref{sub:basic_operator}, we designed another version of WordCount named Windowed Wordcount. Actually, Spark Streaming supports pre-aggregation by default, therefore, above Spark Streaming WordCount experiments already own this feature. Currently, Flink-0.10.1 doesn't support pre-aggregation, and parallel window can only be applied on keyed stream. It is possible to implement Windowed WordCount with Flink's low level API. But it is too time consuming and we leave it to future works. Therefore, only Storm is benchmarked with this workload.

%\begin{figure}
%  \begin{center}
%  \subfigure{\includegraphics[scale=0.35]{images/spark_wordcount_latency}}
%   \caption{Spark Streaming WordCount Latency}
%   \label{fig:spark_wordcount_latency}
%  \end{center}
%\end{figure}

To support Windowed WordCount, we implemented a \texttt{window} operator in Storm\footnote{\url{https://github.com/wangyangjun/Storm-window}} with ack disabled. Our experiments show that the window time has very limited effect on throughput. Here only experiment results of one second window workload are presented. The throughput of Windowed WordCount performing skewed data in Offline model could reach 60K/s (thousand words per second) that is more than two times as large as experiments without window. While dealing with uniform data, the throughput doesn't have any obvious improvement. Pre-aggregation on a window only helps in case of skewed data because it compacts the data thus removing the skew. Online model with a generation speed of 50K/s achieves a median latency of 1431 milliseconds, and a 99-th percentile latency of 3877 milliseconds.

Throughput of WordCount workload is summarized as Table~\ref{table:wordcount_throughput}. Obviously, Flink and Spark Streaming achieve incomparably higher throughput than Storm. The skewness of data has a dramatic effect on WordCount applications without pre-aggregation. 

\begin{table}[H] %\centering
\begin{tabular}{P{2.6cm} | P{2.2cm} | P{2.2cm} | P{2.3cm} | P{2.3cm} } 
% Alignment of sells: l=left, c=center, r=right. 
% If you want wrapping lines, use p{width} exact cell widths.
% If you want vertical lines between columns, write | above between the letters
% Horizontal lines are generated with the \hline command:
\toprule % The line on top of the table
\hline
  & \multicolumn{2}{c|}{No Pre-aggregation}  & \multicolumn{2}{c}{Windowed Pre-aggregation} \\ 
\hline 
 & Uniform & Zipfian & Uniform & Zipfian \\
 \hline
% Place a & between the columns
% In the end of the line, use two backslashes \\ to break the line
% contents of the cell
 Storm  (ack enabled) &  26.6K/s & 21.3K/s &  $\O$  &  $\O$  \\ \hline
 Storm  (ack disabled) &  36.4K/s & 29.4K/s & 50K/s & 60K/s \\ \hline
 Spark Streaming & $\O$ & $\O$ & 1.3M/s & 1.4M/s \\ \hline
 Flink & 2.8M/s & 1.4M/s & $\O$ & $\O$ \\ 
\hline
\bottomrule
\end{tabular} % for really simple tables, you can just use tabular
% You can place the caption either below (like here) or above the table
\caption{WordCount Throughput} 
% Place the label just after the caption to make the link work
\label{table:wordcount_throughput}
\end{table}

\section{AdvClick}

\begin{figure}
  \begin{center}
  \subfigure[Storm]{\label{fig:advclick_sotrm}\includegraphics[scale=0.38]{images/storm_adv_latency}}
  \subfigure[Flink]{\label{fig:advclick_flink}\includegraphics[scale=0.38]{images/flink_adv_latency}}
   \caption{AdvClick Performance}
   \label{fig:adv_click}
  \end{center}
\end{figure}

As described in \cref{subsection:advclick_generator}, click delays of clicked advertisements satisfy normal distribution and the mean is set to 10 seconds. In our experiments, we define that clicks within 20s after corresponding advertisement shown are valid clicks. In theory, overwhelming majority records in the click stream could be joined. Kafka only provides a total order over messages within a partition, not between different partitions in a topic \cite{Kafka}. Therefore, it is possible that click record arrives earlier than corresponding advertisement shown record. We set a window time of 5 seconds for \texttt{advertisement clicks} stream, as acking a tuple would require knowing whether it will be joined with a corresponding one from the other stream in the future.
%Because of these window time, we disabled ack in Storm for this workload.

When benchmarking Storm and Flink, first we perform experiments with low speed data generation, and then increase the speed until obvious joining failures occur when throughput is much less than generation speed of stream \texttt{advertisement clicks}. The experiment result shows that the maximum throughput of Storm cluster is around 8.4K/s (joined events per second). The corresponding generation speed of \texttt{shown advertisements} is 28K/s. As we can see in Figure~\ref{fig:adv_click}\subref{fig:advclick_sotrm}, cluster throughput of  \texttt{shown advertisements} is equal to the data generation speed  when it is less than 28K/s. That means there is no join failures. Figure~\ref{fig:adv_click}\subref{fig:advclick_sotrm} also shows that Storm cluster has a very low median latency. But the 99-th percentile of latency is much higher and increase dramatically with the data generation speed.


%Compared to Storm, Flink achieves a much better throughput. When generation speed of stream \texttt{shown advertisements} increases from 180K/s to 210K/s, Flink cluster stops consuming data from Kafka. The maximum throughput of Flink is between 54 K/s and 63K/s, around 6 times larger than Storm. The latency of Flink performing this workload is shown as Figure~\ref{fig:flink_adv_click}. Even though the median latencies are a little higher than Storm, but 90-th and 99-th percentiles of Flink latency are much lower. 

Compared to Storm, Flink achieves a much better throughput. In our experiments, the throughput of Flink cluster is always equal to the generation speed of stream \texttt{shown advertisements}. But when the generation speed of stream \texttt{shown advertisements} is larger than 200K/s, the Flink AdvClick processing job is usually failed because of a bug in flink-connector-kafka \footnote{\url{https://issues.apache.org/jira/browse/KAFKA-725}}. This issue is fixed in the latest versions of Flink and Kafka. But Storm and Spark don't support the latest version of Kafka yet. We will upgrade all these systems in StreamBench in the next version of StreamBench. The maximum throughput of Flink we achieved in experiments is 63K/s (joined events per second), around 6 times larger than Storm. The latency of Flink performing this workload is shown as Figure~\ref{fig:adv_click}\subref{fig:advclick_flink}. Even though the median latencies are a little higher than Storm, but 90-th and 99-th percentiles of Flink latency are much lower. 

As discussed in \cref{sub:join_operator}, Spark Streaming join operator is applied with sliding window. With the configuration of 20s/5s, the slide intervals of both windows are 5 seconds. That means a micro-batch join job is submitted to Spark Streaming cluster every 5 seconds. Because of different processing model, there is no joining failure in Spark Streaming. But high data generation speed leads to increasing delay of micro-batch jobs, because micro-batch jobs couldn't be finished in interval time. With this configuration, Spark Streaming has a very small throughput which is lower than 2K/s. Increasing micro-batch jobs submitting interval might increase the throughput, but leads to higher latency. For this workload, increasing the window lengths also because of the presence of duplicate records, as the windows overlap. Therefore, we did some experiments with larger windows. Increasing windows length of these two streams to 60s/30s, the cluster could achieve a throughput of 20K/s which is ten times larger. 

\begin{table}[H] %\centering
\begin{tabular}{P{4cm} | P{2cm} | P{2cm} | P{1.8cm} | P{1.8cm} } 
\toprule % The line on top of the table
\hline
   & Maximum  &  \multicolumn{3}{c}{Latency} \\  \cline{3-5}
  & Throughput & Throughout & Median & 90\%  \\ \hline 
 
% Place a & between the columns
% In the end of the line, use two backslashes \\ to break the line
% contents of the cell
 Storm (ack disabled) &  8.4K/s & 4.2K/s & 14ms & 2116ms \\ \hline
 Flink & 63K/s & 33K/s& 230ms & 637ms \\ \hline
 Spark Streaming (20s/5s) & \textless~2K/s & $\O$ &  $\O$  & $\O$  \\ \hline
 Spark Streaming (60s/30s) & 20K/s & 20K/s & $\sim$20s & $\sim$24s  \\

\hline
\bottomrule
\end{tabular} % for really simple tables, you can just use tabular
% You can place the caption either below (like here) or above the table
\caption{Advertisement Click Performance} 
% Place the label just after the caption to make the link work
\label{table:AdvClick}
\end{table}

Table~\ref{table:AdvClick} summarizes maximum throughputs and latencies at a specific throughput of these systems. Flink achieves the largest throughput and lowest 90-th percentile latency. While the median latency of Storm is 14ms, that is much lower than other systems. Latencies of Spark Streaming shown in the table is the latencies of micro-batches that doesn't include buffer time of records in a window.

\section{K-Means}

Experiment results of stream k-means processing 2-dimensional points shows that Storm cluster with at-least-once processing guarantee has a throughput of 1.7K/s. Without this guarantee, the throughput is a litter higher, around 2.7K/s. The maximum throughput of Flink cluster is much larger and reaches 78K/s. Figure~\ref{fig:kmeans_latency} shows the latencies of Flink cluster and Storm Cluster without at-least-once guarantee. When the generation speed of point stream is low, Storm achieves very low median latency. The 90-th percentile of latency of Storm is also lower than Flink. The latency of Storm rises sharply when the generation speed is around 2.7K/s to 3K/s. Compared with Storm, latency percentiles of Flink is more compact. When the speed of data generations is 30K/s, Flink achieves a median latency of 141 milliseconds, and a 90-th percentile latency of 195 milliseconds. From this figure, it is easy to know that the throughput of Flink is significantly larger than storm.

%Figure~\ref{fig:kmeans_latency} shows the latencies of Storm and Flink clusters operating at around half size of corresponding max throughput. Both Storm with and without at-least-once guarantee achieve very low median latency and a much higher 99-th percentile latency. Generally, the latency of Storm with at-lest-once guarantee is a little higher. Compare with Storm, latency percentiles of Flink is more compact. Flink achieves a median latency of 122 milliseconds, and a 99-th percentile latency of 310 milliseconds. 
 
\begin{figure}
  \begin{center}
  \subfigure{\includegraphics[scale=0.5]{images/kmeans_flink_storm_latency}}
   \caption{KMeans Latency of Flink and Storm}
   \label{fig:kmeans_latency}
  \end{center}
\end{figure}

% convance
Since the throughput of Flink is tens of times higher than Storm, this workload converges much quicker on Flink cluster. In order to compare convergences of the algorithm running on Storm and Flink clusters, we calculated average distance between centroids and corresponding nearest center over the number of points processed on each compute node and visualized as Figure~\ref{fig:converge}\subref{fig:flink_storm}. The results indicate that the k-means algorithm performing on Storm cluster achieves a little better convergence.

\begin{figure}[t!]
  \begin{center}
   \subfigure[Flink and Storm]{\label{fig:flink_storm}\includegraphics[scale=0.4]{images/converge}}
  ~
  \subfigure[Spark Streaming]{\label{fig:spark}\includegraphics[scale=0.4]{images/converge_spark}}
   \caption{Convergences}
   \label{fig:converge}
  \end{center}
\end{figure}

Because of Spark's DAG computational model, Spark Streaming doesn't support iterate operator. Instead of forwarding updated centroids in a nested cycle, Spark Streaming implements stream k-means in a different way. It maintains a clustering model and updates the model after each micro batch processed. The update interval of clustering model is the same as micro batch interval. More detail about Spark Streaming K-Means could be found online\footnote{\url{https://databricks.com/blog/2015/01/28/introducing-streaming-k-means-in-spark-1-2.html}}. With the default setting of one second micro-batch interval, to keep  the average latency of micro-batches is less than one second, the maximum throughput of the cluster achieved is around 1M/s. The average latencies of micro-batches processing data with different generation speed are shown as Figure~\ref{fig:spark_kmeans_latency}. The latency of a micro-batch is the time from when the batch is ready to the time that the processing job is done. The latency of each record also includes the time that the record buffered in a window.
The experiment results show that the convergence of Spark Streaming K-Means is very fast. Usually, it is converged in a few micro-batches. Figure~\ref{fig:converge}\subref{fig:spark} shows the convergence of Spark Streaming over the number of processed events. It is obvious that stream with lower speed converges faster. But stream with higher speed achieves lower average distance between centroids and real centers after converged. In Figure~\ref{fig:converge}, it is easy to notice that the workload running in Flink and Storm is converged after 2000 points processed, that is much lower than Spark Streaming. This is mainly because of the difference between processing models of these systems. In Flink and Storm, record in a stream is processed one by one. That means once a latest centroid is calculated, it could be updated to the k-means clustering model. While in Spark Steaming, the k-means cluster model is updated when a micro-batch job is done. The update frequency is significantly lower than Flink and Storm.
 
%spark latency
\begin{figure}
  \begin{center}
  \subfigure{\includegraphics[scale=0.24]{images/spark_kmeans_latency}}
   \caption{Spark KMeans Latency}
   \label{fig:spark_kmeans_latency}
  \end{center}
\end{figure}

\begin{table}[H] %\centering
\begin{tabular}{P{2.2cm} | P{2.2cm} | P{2cm} | P{1.5cm} | P{1.5cm} | P{1.5cm}} 
\toprule % The line on top of the table
\hline 
   & Maximum  &  \multicolumn{3}{c}{Latency} \\  \cline{3-6}
  & Throughput (K/s) & Throughout (K/s) & Median (ms) & 90\% (ms)& 99\% (ms) \\ \hline 
 Storm (ack enabled) &  1.7 & 0.9 & 13 & 100 & 410 \\ \hline
 Storm (ack disabled) &  2.7 & 1.6 & 21 & 107 & 388 \\ \hline 
 Flink & 78 & 40 & 122 & 183 & 310 \\ \hline
 Spark Streaming & 1000 & 480 & 986 & 1271 & 1837 \\

\hline
\bottomrule
\end{tabular} % for really simple tables, you can just use tabular
% You can place the caption either below (like here) or above the table
\caption{KMeans Performance} 
% Place the label just after the caption to make the link work
\label{table:kmenas}
\end{table}

The performance of this workload is summarized in Table~\ref{table:kmenas}. It is clear that Spark Streaming achieves the best maximum throughput, and Storm achieves the lowest median latency. The percentile latencies of Flink is more compact than that of Storm. We also performed some experiments with high dimension point stream. The experiment results show that increasing point dimension has very limit effect on workload performance. The main result of increasing point dimension is leading to more computation in point distance calculation. Which indicates that computation is not a bottleneck of this workload. 
%Compared with low point dimension, high dimension leads to larger record size and more computation in point distance calculation. Which indicates that computation is not a bottleneck of this workload. 

\section{Summary}

The experiment results of these workloads show that both Flink and Spark Streaming achieve significantly higher throughput than Storm. But Storm usually achieves a very low median latency. Median latencies of these workloads running in Flink is much higher than Storm. But Flink achieves a similar 99-th percentile latency. In most case, the latencies of Storm and Flink is less than one second. Compared with other two workloads, all these systems get a worse performance in workload AdvClick. Because of micro-batch computational model, there is a tradeoff between throughput and latency of Spark Streaming executing this workload. Normally, the latency of Spark Streaming is much larger than Storm and Flink. 

In practice, the selection of stream processing systems depends on the situation. If an application requests very low latency, but the requirement of throughput is not stringent, Storm would be the best choice. On the contrary, if throughput is the key requirement, Spark Streaming is a very good option. Flink achieves both low latency and high throughput. For most stream processing cases, Flink would be a good choice. 

\clearpage




 
\chapter{Conclusions}
Summary of experiment results 

\section{Selection in Practice}
Summarize several factors which affect selection of stream processing systems in practic

\subsection{Performance Summary}
\subsection{Issues }

\section{Future Work}
Future works

The amount of data in Kafka affects the performance of Offline WordCount, especially for Storm.
Increase computation nodes, to test scaleability. 
Benchmark more stream processing systems and design more workloads.

\subsection{Scale-out and Elasticity Evaluation}
\subsection{Evaluation of Other Platforms}

\clearpage

\phantomsection
\addcontentsline{toc}{section}{\refname}

% Load the bibliographic references
% ------------------------------------------------------------------
% You can use several .bib files:
% \bibliography{thesis_sources,ietf_sources}
\bibliographystyle{IEEEtran}
\bibliography{sources}
\clearpage


%% Appendices
%% Liitteet
\thesisappendix
\section{Source Code\label{LiiteA}}
\subsection{WordCount}
\subsection{Advertisements Click}

\clearpage




\end{document}
